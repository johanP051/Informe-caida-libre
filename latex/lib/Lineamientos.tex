\begin{document}

Presentación:\\

El trabajo debe ser simplemente grapado, sin hoja de respeto en blanco. Máximo 3 páginas, tamaño de letra 10(pt) (según sea el caso o (9pt a 8pt)). Márgenes de 2,0 cm en cada extremo. El texto puede estar en una o dos columnas. No debe anillarse ni utilizar fólder o clip.\\

Nota: A los documentos no se les calificará por el tamaño del marco teórico.\\

Recomendaciones adicionales\\

Para la escritura del informe se debe utilizar una buena redacción. Se deben poner en práctica las reglas gramaticales.\\

Cuando se redacta un documento se requiere:\\

*precisión,
*claridad y
*brevedad.

Use palabras que comuniquen exactamente lo que se quiere decir (precisión). Escriba en forma clara, fácil de entender y con un lenguaje sencillo (claridad). Identifique el texto innecesario y evítelo, ya que desvía la atención del lector y en ocasiones dificulta la claridad de lo que quiere expresarse. Tenga en cuenta que las oraciones muy largas son difíciles de entender, o un párrafo que ocupa una página completa abruma al lector y no invita a la lectura; si es necesario desarrollar varias ideas use varias frases para expresar cada una de ellas (brevedad).\\

Al escribir el informe, recuerde que "una imagen vale más que mil palabras", utilice dibujos, esquemas o ilustraciones eficaces. Póngalos en el lugar en que el texto las menciona y explíquelos.\\

Algunos errores frecuentes presentes en los informes se enumeran a continuación, procure evitarlos.\\

Falta de orden en el planteamiento de las ideas: se salta de una idea a otra y se vuelve a la inicial, incluso repitiendo conceptos ya expresados. Lo que se recomienda es hacer primero un esquema teórico con los títulos, tratando que cada enunciado responda una pregunta específica y no mezclar varios asuntos al mismo tiempo.\\

Errores de Ortografía: Acentuar inadecuadamente las palabras (pérdi-da por perdida, gráfica por grafica), confundir letras cuyos fonemas son similares (vaya por baya o valla) y otras. Aproveche el auto-corrector del computador y la posibilidad que tiene de consultar los diccionarios en línea (http://lema.rae.es/drae/?). Palabras que están mal escritas, llevan al lector a interpretaciones que pueden ser contrarias a lo que el escritor quiere expresar.\\

Falta o mal uso de la coma, el punto y coma, el punto seguido o el punto final: poner en un lugar inadecuado un signo de puntuación cambia completamente la idea que se quiere comunicar ("Si Juan supiera el valor que tiene su novia, le besaría los pies" por "Si Juan supiera el valor que tiene, su novia le besaría los pies").

Falta de coherencia en el uso de los tiempos verbales: empezar, por ejemplo, una frase en presente, seguir luego conjugando los verbos en pasado o en futuro en el mismo párrafo.\\

Copiado literal (copy-paste): plagio, fraude; esta falta consiste en reescribir completamente lo dicho en otro texto sin aclarar la fuente de la que se ha tomado, dando la falsa impresión de que usted es el autor. Si necesariamente tiene que usar lo escrito en otro texto debe ponerlo entre comillas; inmediatamente, antes o después de la cita, debe aclarar la referencia de la cual la ha tomado. Lo más recomendable es que usted lea los textos necesarios y exprese con sus propias palabras las ideas relevantes.\\

Falta de corrección autocrítica: después de terminar un escrito, es difícil que el mismo autor vea más errores de los que ya detectó. El cerebro se niega a ver cosas nuevas. Un lector imparcial, con cierta formación, se percatará de ellos al instante. El cansancio, la premura por entregar, la falta de documentación o de rigor o hasta la vanidad\\

personal influyen muchas veces en ello. Es recomendable “dejar enfriar” el escrito unos días para corregirlo, por lo tanto el informe debe realizarse días antes de su entrega y todos los miembros del grupo deben leer la totalidad del documento.\\

Falta de concordancia con la bibliografía: se escribe sobre un tema y la bibliografía citada es de otro diferente, sin nada que ver con el asunto, o peor aún: al buscar la referencia no existe, dando la impresión que hay desinterés por el documento que se escribe o que la intención sea llenar por llenar.\\

Tenga en cuenta que para justipreciar los informes se dará mayor valor a los escritos que conservan:\\

La rigurosidad (exactitud, precisión y minuciosidad) en la metodología del trabajo. La calidad en la redacción (ortografía y estilo), así como la sobriedad en la presentación y la conciencia ecológica.

\end{document}
